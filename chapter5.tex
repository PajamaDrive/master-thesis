\documentclass[a4paper,11pt]{jreport}
\usepackage[master]{tuats}
\usepackage{ascmac}
\usepackage{mathtools}
\usepackage{booktabs}
\debugtrue

% PDF用メタデータ
% コメントアウトすれば著者名やタイトルをPDFに埋められるが,問題が発生しがち
% \AtBeginDvi{
%     \special{pdf:tounicode EUC-UCS2}
%     \special{pdf:docinfo <<
%         /Author (安井貴規 (Takaki YASUI))
%         /Title  (藤田桂英研究室栄光のLaTeXフォーマット (A LaTeX Style Files for Fujita Katsuhide Labratory))
%     >>}
% }

%タイトル
\title{人間とエージェントの交渉における \\ プロファイリングを用いた交渉戦略}
\etitle{Strategy using profiling for human-agent negotiation}

%名前
\author{松下昌悟}
\eauthor{Shogo Matsushita}

%入学年度
\enteryear{2019}
%卒業年度
\graduateyear{2021}

%学籍番号
\studentnumber{19646029}

%提出日
\date{2021年1月X日}
\begin{document}

\chapter{プロファイリング決定におけるパラメータ決定のための予備実験}
\chapref{chap:profiling}で述べたビッグファイブの測定に関して,各種パラメータの値を調節することを目的として人間の被験者に対して実験を行う.
同時に,エージェントの目標効用に関するパラメータが適切であるかを評価する.
予備実験の結果によってエージェントのパラメータ調節を行い,評価実験を実施する.

\section{実験設定}
被験者はIAGOを用いてエージェントと交渉を行う.
被験者はエージェントと交渉を行う前にチュートリアルを行い,あらかじめIAGOの操作に慣れてから実験を行った.
予備実験では,ドメインとしてFavorGameSpec1を使用する.
また,被験者に対して事前質問を行い被験者のビッグファイブを測定する.
事前質問による値と交渉中に測定した値を比較することで人間のビッグファイブの値を測定できているか評価する.
なお,事前質問には50項目の質問によってビッグファイブの値を測定するIPIP尺度を利用する.
各項目は``まったく当てはまらない"から``よく当てはまる"までの5段階評価となっており,
各項目の評価値によってビッグファイブの値を測定する.

予備実験に用いるエージェントは,交渉中に測定したビッグファイブの値を使用せず,事前質問によるビッグファイブの値を目標効用の計算に用いる.
すなわち,TKIの積極性,協調性の値は交渉中に計算するがビッグファイブの値を固定値として目標効用を計算する.
これにより,目標効用のパラメータ設定が測定したビッグファイブの値に対して適切に行われているかを確認する.
予備実験の被験者は3人であり,いずれも弊学の学生である.

\section{パラメータの設定}
予備実験ではTKIの協調性,積極性,ビッグファイブの各因子の計算の途中で用いるパラメータ$N$,$\psi$,$\nu$を\tabref{tab:pre_other-param},$\xi$を\tabref{tab:pre_xi-param}に示す値に設定する.
これらの値はあらかじめヒューリスティックに決定したものである.
\begin{table}[!tb]
    \centering
    \caption{$N$,$\psi$,$\nu$の値}
    \begin{tabular}{lll} \toprule
        \begin{tabular}{c}$bids$に関する\\パラメータ\end{tabular} & \begin{tabular}{c}$behs$に関する\\パラメータ\end{tabular} & \begin{tabular}{c}$\mathit{offs}$に関する\\パラメータ\end{tabular} \\ \midrule
        $N_{bid} = 3$ & $N_{beh} = 10$ & $N_{\mathit{off}} = 5$ \\
        $\psi_{bid}^{pre} = 0.1$ & $\psi_{beh}^{pre} = 0.5$ & $\psi_{\mathit{off}}^{pre} = 1.0$ \\
        $\nu_{bid}^{lim} = 0.5$ & $\nu_{beh}^{lim} = 0.5$ & \\ \bottomrule
    \end{tabular}
    \label{tab:pre_other-param}
\end{table}

\begin{table}[!tb]
    \centering
    \caption{$\xi$の値}
    \begin{tabular}{llll} \toprule
        因子 & \begin{tabular}{c}最小値の計算に\\用いるパラメータ\end{tabular} & \begin{tabular}{c}最大値の計算に\\用いるパラメータ\end{tabular} & \begin{tabular}{c}その他の\\パラメータ\end{tabular} \\ \midrule
        TKI & & & \\
        協調性 & $\xi_{min}^{util} = 0.45$ & $\xi_{max}^{util} = 0.65$ \\
        積極性 & $\xi_{min}^{dev} = 0.4$ & $\xi_{max}^{dev} = 0.8$ \\ \midrule
        ビッグファイブ & & & \\
        神経症傾向 & $\xi_{min}^{\mathit{offN}} = 0.1$ & $\xi_{max}^{\mathit{offN}} = 0.75$ & $\xi^{Nice} = 1.0$ \\
        & $\xi_{min}^{\mathit{behN}} = 0.0$ & $\xi_{max}^{\mathit{behN}} = 7.5$ & $\xi^{Concession} = 1.0$ \\
        外向性 & $\xi_{min}^{\mathit{offE}} = 0.1$ & $\xi_{max}^{\mathit{offE}} = 0.75$ & $\xi^{\mathit{Selfish}} = 1.0$ \\
        & $\xi_{min}^{\mathit{freq}} = 10.0$ & $\xi_{max}^{\mathit{freq}} = 40.0$ & $\xi^{Fortunate} = 1.0$ \\
        & $\xi_{min}^{\mathit{favReq}} = -1.0$ & $\xi_{max}^{\mathit{favReq}} = 1.0$ \\
        経験への開放性 & $\xi_{min}^{\mathit{devO}} = 0.2$ & $\xi_{max}^{\mathit{devO}} = 0.675$ & $\xi^{batAsk} = 2.0$ \\
        & $\xi_{min}^{\mathit{behO}} = 0.0$ & $\xi_{max}^{\mathit{behO}} = 7.5$ \\
        協調性 & $\xi_{min}^{\mathit{sense}} = 0.5$ & $\xi_{max}^{\mathit{sense}} = 1.75$ \\ 
        & $\xi_{min}^{\mathit{ratioA}} = 0.2$ & $\xi_{max}^{\mathit{ratioA}} = 0.7$ \\
        & $\xi_{min}^{\mathit{behA}} = 0.0$ & $\xi_{max}^{\mathit{behA}} = 7.5$ \\
        & $\xi_{min}^{\mathit{favRet}} = -1.0$ & $\xi_{max}^{\mathit{favRet}} = 1.0$ \\
        誠実性 & $\xi_{min}^{\mathit{ratioC}} = 1.0$ & $\xi_{max}^{\mathit{ratioC}} = 7.0$ & $\xi^{batTel} = 2.0$ \\
        & $\xi_{min}^{\mathit{behC}} = 0.0$ & $\xi_{max}^{\mathit{behC}} = 7.5$ \\
        & $\xi_{min}^{\mathit{fastBeh}} = 0.0$ & $\xi_{max}^{\mathit{fastBeh}} = 10.0$ \\
        & $\xi_{min}^{\mathit{lie}} = -1.0$ & $\xi_{max}^{\mathit{lie}} = 1.0$ \\ \bottomrule
    \end{tabular}
    \label{tab:pre_xi-param}
\end{table}

\section{実験結果と考察}
\subsection{交渉結果}
エージェントと人間の効用の平均を\figref{fig:pre_util}に示す.
なお,黒の破線は効用値0.7を表しており,図中のエラーバーは95\%信頼区間である.

\begin{figure}[bt]
    \centering
    \includegraphics[width=9truecm]{../image/pre_utility.eps}
    \caption{エージェントと人間の個人効用}
    \label{fig:pre_util}
\end{figure}

\figref{fig:pre_util}のように,エージェントの効用の平均は人間の効用の平均より高い値となっている.
また,いずれの被験者に対してもエージェントは0.7以上の高い効用を獲得することができた.
そのため,目標効用に関するパラメータ設定は妥当であると考えられる.
また,$\alpha = 0.05$として対応のある$t$検定を行った結果,$p \geqq 0.05$であり,有意差はなかった.
この実験はサンプル数が3と非常に少なかったことが要因として考えられる.

\subsection{被験者のビッグファイブの推定}
被験者のビッグファイブの値を\figref{fig:pre_big5}に示す.
青のチャートは事前質問により測定した値,橙のチャートは交渉中に計算した各因子の平均値,緑のチャートは交渉終了時に計算した各因子の値である.

\begin{figure}[bt]
    \begin{minipage}[b]{0.47\linewidth}
        \centering
        \includegraphics[width=7.5truecm]{../image/raderChart_pre_1.eps}
        \subcaption{被験者A}\label{fig:pre_big5_sub1}
    \end{minipage}
    \begin{minipage}[b]{0.47\linewidth}
        \centering
        \includegraphics[width=7.5truecm]{../image/raderChart_pre_2.eps}
        \subcaption{被験者B}\label{fig:pre_big5_sub2}
    \end{minipage}\\
    \begin{center}
        \begin{minipage}[b]{0.47\linewidth}
            \centering
            \includegraphics[width=7.5truecm]{../image/raderChart_pre_3.eps}
            \subcaption{被験者C}\label{fig:pre_big5_sub3}
        \end{minipage}
    \end{center}
    \caption{被験者のビッグファイブの値}\label{fig:pre_big5}
\end{figure}

\figref{fig:pre_big5}より,いずれの被験者も交渉中に計算した神経症傾向の値が事前質問の値よりも非常に低くなった.
神経症傾向はネガティブな情動を引き起こすが,その情動は内側,すなわち自己に向きやすい.
また,ネガティブな行動というのは一般的に好ましくない.
したがって,ネガティブな行動は自らがマイナスであるという印象を交渉相手に与え,それが交渉結果にも影響する可能性がある.
これらを考慮した結果,ネガティブな行動そのものが交渉においてあまり行われなかったと考えられる.
そのため,交渉相手に対してはネガティブな行動を行う回数が少なく,神経症傾向の値を十分に測定できなかった可能性が考えられる.

一方で,外向性の値は事前質問の値よりも非常に高くなった.
交渉は合意形成を行うことが主目的であるが,その中で,ほとんどの交渉者は自分の利益を少しでも多く得ようとすると推測できる.
また,報酬に対する欲求が非常に高いと外向性は高くなるため,交渉という行為のみでビッグファイブの測定を行うと外向性が高くなりやすいと考えられる.

協調性については事前質問の値と計算値が同じような値となっている.
協調性は提案の効用値の割合を測定に用いているが,
このような結果から自分勝手な提案を行う人は予想より少ないと考えられる.
また,ポジティブな行動は一般的に好ましい.
したがって,ポジティブな行動はネガティブな行動よりも多く行われ,測定が容易だったと考えられる.

これに対して,経験への開放性と誠実性に関してはあまり傾向が掴めなかった.
これら2つの因子の測定には選択肢,効用の標準偏差を使用している.
そのため,測定のぶれが大きく正しく測定ができなかった.

また,ビッグファイブの測定に関して提案の内容を重視しすぎた可能性がある.
IAGOにおいて提案以外の行動は必須ではない.
したがって,提案のみを行う可能性もあるため,常に$\nu_{\mathit{bid}} \geqq \nu_{\mathit{beh}}$としていた.
しかし,提案の内容のみでビッグファイブの各因子を正確に測定するのほ困難である.
よって,ビッグファイブの測定がうまく行えなかったと考えられる.

予備実験では,3人の被験者がエージェントと交渉を行ったが,いずれの被験者も交渉に関する研究を行っていた.
そのため,交渉について熟知しており,合意形成に至るまでのラウンド数が非常に少なかった.
また,選好をエージェントに聞くのではなく,提案内容からおおよその選好を推測する傾向があった.
これらの要因により,エージェントが相手の特性を推定するために必要な情報を十分に収集することができなかった可能性がある.

\bibliography{reference}
% 付録用
%\chapter*{付録}


\ifthmaster
  \externals
\begin{externalsenum}{H}
\item \underline{松下昌悟},藤田桂英.
    心理的効果を用いた人間とエージェントの繰り返し交渉戦略.
    電子情報通信学会 人工知能と知識処理研究会(AI), July 2019.
\end{externalsenum}

\fi

\end{document}