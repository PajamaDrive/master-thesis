
交渉は要求や意見の異なる他者と共同で意思決定を行うプロセスであり,社会生活を営む上で必要不可欠な工程である.
しかし,大学を卒業した者であっても交渉スキルは十分でなく,企業は交渉スキルの向上に高いコストをかけている.
エージェントを用いることで低いコストで交渉スキルの向上が行えるため,人間とエージェントの交渉に関心が高まっている.

人間とエージェントの交渉ではTKIを用いて相手の特性を推定し,譲歩速度を変更するエージェントが良い交渉結果を得ている.
しかし,部分的な提案を許可していない,提案内容のみで相手の特性を推定している,TKIでは相手の特性を的確に推定することは困難であるなどの
問題点がある.
人間とエージェントの交渉では提案内容以外にも感情やメッセージの送信が可能であり,これはエージェント同士の交渉である自動交渉と大きく異なる点である.
しかし,提案以外の行動を考慮せずに戦略を決定している研究がほとんどである.

本稿ではTKIとビッグファイブを併用して相手の特性を推定し,譲歩速度を変更するエージェントを提案する.
また,相手の感情やメッセージなどの提案以外の行動と提案内容からビッグファイブの各因子の値を測定することを試みる.
推定したTKIの各モードに分類される確率,ビッグファイブの各因子の値によってエージェントの譲歩速度を変更する.
これにより,相手の特性に合わせて提案を行うことができるエージェントを実現する.

ビッグファイブの測定に用いるパラメータを調整するために予備実験を行い,
予備実験により決定したパラメータを用いて評価実験を行った.
評価実験により,ビッグファイブを用いた戦略は社会的余剰が高く,公平な解に到達可能であり,有効性が高いことが示された.
また,相手に悪い印象を与えることなく相手の譲歩を引き出すことができ,より現実的な提案を行うことが可能であると示された.
