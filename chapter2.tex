\documentclass[a4paper,11pt]{jreport}
\usepackage[master]{tuats}
\usepackage{ascmac}
\usepackage{mathtools}
\usepackage{booktabs}
\debugtrue

% PDF用メタデータ
% コメントアウトすれば著者名やタイトルをPDFに埋められるが,問題が発生しがち
% \AtBeginDvi{
%     \special{pdf:tounicode EUC-UCS2}
%     \special{pdf:docinfo <<
%         /Author (安井貴規 (Takaki YASUI))
%         /Title  (藤田桂英研究室栄光のLaTeXフォーマット (A LaTeX Style Files for Fujita Katsuhide Labratory))
%     >>}
% }

%タイトル
\title{人間とエージェントの交渉における \\ プロファイリングを用いた交渉戦略}
\etitle{Strategy using profiling for human-agent negotiation}

%名前
\author{松下昌悟}
\eauthor{Shogo Matsushita}

%入学年度
\enteryear{2019}
%卒業年度
\graduateyear{2021}

%学籍番号
\studentnumber{19646029}

%提出日
\date{2021年1月X日}
\begin{document}

\chapter{関連研究}
\label{chap:related}
\section{自動交渉エージェント競技会(ANAC)}

自動交渉エージェント競技会(Automated Negotiation Agent Competition)は各自が作成したエージェント同士で交渉を行い
個人獲得効用や社会的余剰などを競う競技会である\cite{anac2010-2017,anac2018,anac2019,anac2020}.
ANACは毎年開催され,2010年(第1回)から2016年(第7回)まではInternational Conference on Autonomous Agents and Multiagent Systems(AAMAS)と共催で行われた.
2017年からはInternational Joint Conference on Artificial Intelligence(IJCAI)と共催で行われ,
2020年(第11回)もIJCAIと共催で行われる予定である.
ANACは以下の4点を目的としている.

\begin{itemize}
    \item 未知の相手に対し,様々な状況で巧みに交渉できる実用的な交渉エージェントの設計を促進する
    \item 様々な交渉戦略を客観的に評価する指標を提供する
    \item 様々な学習手法,適応戦略,交渉相手のモデル構築を探求する
    \item 最先端の交渉エージェントと交渉シナリオを収集し,研究コミュニティに提供する
\end{itemize}

ANACでは2016年までは自動交渉のプラットフォームであるGeniusを用いて単一のリーグが開催されていた.
2017年からは複数のリーグが開催され,2020年は以下の5つのリーグが開催される予定である.

\begin{itemize}
    \item Automated Negotiation League(エージェント同士での交渉)
    \item Human-Agent League(人間とエージェントの交渉)
    \item Supply Chain Management League(サプライチェーンにおける交渉)
    \item Werewolf Game League(人狼ゲームにおける交渉)
    \item HUMAINE(対話形式による人間と複数エージェントとの交渉)
\end{itemize}

\section{IAGO}
IAGO(Interactive Arbitration Guide Online)はANACのHuman-Agent Negotiationリーグで使用される自動交渉のプラットフォームである\cite{iago}.
IAGOのインタフェースを\figref{fig:iago}に示す.

\begin{figure}[tb]
    \centering
    \includegraphics[width=15truecm]{../image/IAGO.eps}
    \caption{IAGOのインタフェース}
    \label{fig:iago}
\end{figure}

IAGOでは,人間と交渉することが可能なエージェントおよび人間とエージェントの交渉に用いるドメインを作成することが可能である.
IAGOの主な特徴として以下の3点が挙げられる\cite{pinocchio}.

\begin{enumerate}
    \item 作成したエージェントはWEBサーバ上で動作し,ユーザはソフトウェアのインストールなどを行うことなくWEBブラウザ上で簡単にエージェントと交渉を行うことができる
    \item エージェントおよび交渉ドメインの設計・開発を独立して行えるようにAPIが展開されている
    \item 人間同士の交渉で用いられる行動・チャネルが複数使用可能である
\end{enumerate}

\secref{sec:introduction}で述べたように自動交渉の開発を行うプラットフォームとしてGenius\cite{genius}やNegMAS\cite{negmas}などがあるが,
これらのプラットフォームではエージェント同士での交渉を前提としているため,感情やメッセージの送受信など交渉で人間が行う行動を取り扱うことができない.
一方,IAGOは人間と交渉できるエージェントを開発し,研究・シミュレーションを行うことができるように設計されている.
そのため,IAGOで行われる交渉において人間およびエージェントは他のプラットフォームではサポートされていない様々な行動を行うことが可能である.
交渉者は部分的な提案を行うことが可能で,さらに提案だけでなく感情の表出やメッセージの送信,お互いの選好に関する情報を送受信することができる.
以上のような人間同士の交渉で用いられる諸要素をユーザおよびエージェントが使用することができるため,
実際に行われる交渉と近い状況で人間との交渉をシミュレーションすることができる.

このように,人間と交渉可能なエージェントを開発することができるため,IAGOを用いて実験を行う研究が増えつつある.
また,最新版のIAGOでは留保価格についての情報交換や見返りの要求や約束が行えるようになっており,より現実に即した交渉を行うことが可能になった.
最新版のIAGOを用いて見返りを求めるエージェントの方が高い効用を得るという結果が示唆されている\cite{favor}.

\section{TKI}
\label{sec:tki}
TKIは対立した状況における個人の行動を評価・分析するための指標である\cite{tki}.
TKIでは協調性(cooperativeness)と積極性(assertiveness)の2つの軸を用いる.
協調性は相手の利益や要求を満たそうとする程度を表し,積極性は自分自身の利益や要求を満たそうとする程度を表す.
TKIでは協調性と積極性の値によって以下の5つのモードに分類される.
\begin{description}
    \item{\textbf{共存(Collaborating)}}\mbox{}\\
    協調性が高く,積極性も高い人は共存モードに分類される.
    共存モードは相手と一緒に互いの要求を満たすような解決策を探る.
    また,両者の根本的な要求を把握し,両者の要求を満たすような代替案を探るために問題を深掘りする.
    \item{\textbf{順応(Accommodating)}}\mbox{}\\
    協調性が高く,積極性が低い人は順応モードに分類される.
    順応モードは親身であり,自分の利益を無視してでも相手を満足させる.
    このように順応モードは自己犠牲の精神があり,相手の命令や脅迫に従うことがある.
    \item{\textbf{競争(Competing)}}\mbox{}\\
    協調性が低く,積極性が高い人は競争モードに分類される.
    競争モードは衝突が起こった際,相手を犠牲にして自分の利益を追求する.
    また,自分の権利のために立ち上がり,自分が正しいと信じた信念を守り,相手に勝利するために行動する.
    \item{\textbf{回避(Avoiding)}}\mbox{}\\
    協調性が低く,積極性も低い人は回避モードに分類される.
    回避モードは自分自身または相手の利益を追求することはない.
    また,自ら問題に対処することはなく,問題のすり替え,問題への対処を延期,問題からの撤退などの形を取る可能性がある.
    \item{\textbf{妥協(Compromising)}}\mbox{}\\
    協調性と積極性が中庸な人は妥協モードに分類される.
    妥協する際の目標は両者が部分的に満足でき,互いに受け入れることができるような内容のものである.
    妥協の度合いとしては競争モードよりは譲歩し順応モードよりは譲歩しない.
    また,回避モードよりは問題に対処しようとするが共存モードのように問題を深掘りするわけではない.
    このように妥協モードはすべてのモードの中間の行動をとる.
\end{description}

TKIは上述のように各モードに典型的な特徴があり,その上で\figref{fig:tki}のように個人がどのモードを使用しやすいかを分類することで個人の特徴を判断する.
\begin{figure}[tb]
    \centering
    \includegraphics[width=11truecm]{../image/tki.eps}
    \caption{TKIによるパーソナリティの表現}
    \label{fig:tki}
\end{figure}

\section{パーソナリティに関する研究}
\label{sec:personality}
本稿で用いるビッグファイブは人間のパーソナリティを5つの因子で表現するものである.
まず,ビッグファイブが提唱されるまでのパーソナリティ研究について簡単にまとめる\cite{big5-history, psychology, first-personality,daniel}.

\subsection{類型論}
\secref{sec:tki}で述べたTKIのように人間のパーソナリティをいくつかの大きなグループに分類し,
グループごとに典型的な性格や特徴などを記述する手法を類型論もしくは類型説と呼ぶ.
類型論的なパーソナリティの記述は古代から行われている.
人間の性格について記述された現存する世界最古の書物は,哲学者であり植物学者でもあるテオプラストスによって紀元前に書かれた``人さまざま"であるとされている.
この書物の中ではへそまがり,お節介,傲慢などの30種類のパーソナリティを定義し,その特徴や具体的な人物像について記述されている\cite{theophrastus}.
また,テオプラストスが誕生する以前にヒポクラテスは人間の4種類の体液のバランスの歪みによって病気が生じるとする四体液説を提唱した\cite{hippocrates}.
ここで,4種類の体液は血液,粘液,黒胆汁液,黄胆汁液であり,各体液は季節や四大元素などと対応している.
ヒポクラテスの四体液説から500年後,ガレノスが四体液説を発展させ,四気質説という類型論を展開し,後世に大きな影響を与えた\cite{smith}.
四気質説および四体液説の内容を\tabref{tab:galenos}に示す.

\begin{table}[tb]
    \centering
    \caption[ガレノスによる四気質説とヒポクラテスの四体液説]{ガレノスによる四気質説とヒポクラテスの四体液説 \protect \footnotemark}
    \begin{tabular}{lllll} \toprule
        四気質説 & & & 四体液説 & \\ \midrule
        体液 & 気質 & 特徴 & 季節 & 元素\\ \midrule
        血液 & 多血質 & 怒りっぽい,衝動的,楽天的 & 春 & 空気 \\
        粘液 & 粘液質 & 受動的,用心深い,考え深い & 冬 & 水 \\
        黒胆汁液 & 黒胆汁質 & 心配性,悲観的,内気 & 秋 & 土 \\
        黄胆汁液 & 黄胆汁質 & 愛想が良い,社交的,陽気 & 夏 & 火 \\ \bottomrule
    \end{tabular}
    \label{tab:galenos}
\end{table}
\footnotetext{出典: 小塩. \cite{first-personality}(一部修正)}

その後,ガレノスの説いた四気質説は人間の性格を記述する際の基本的な考えとなった.
例えば,Kantはガレノスの気質論を踏襲しつつ新たな解釈を加えた\cite{kant}.
Kantは人間の気質はまず感情と活動の2つの気質に分類でき,さらにそれぞれの下位分類として生命力の興奮と弛緩の2種類があり,
最終的に人間をこれら4つの気質に分類できると述べた.
\tabref{tab:kant}に示すようにKantは多血質と憂鬱質は感情の気質,胆汁質と粘液質は活動の気質であるとした.

\begin{table}[tb]
    \centering
    \caption{Kantによる気質の分類}
    \begin{tabular}{llll} \toprule
        上位分類 & 下位分類 & 気質 & 特徴 \\ \midrule
        \multirow{2}{*}{感情} & 興奮 & 多血質 & 無頓着,社交的,機嫌が良い \\
        & 弛緩 & 憂鬱質 & 慎重,悲観的,物事を重く受け止める \\ \midrule
        \multirow{2}{*}{活動} & 興奮 & 胆汁質 & 熱血漢,欲深い,プライドが高い \\ 
        & 弛緩 & 粘液質 & 情動があまり変動しない,持続的,ずる賢い \\ \bottomrule
    \end{tabular}
    \label{tab:kant}
\end{table}

パーソナリティ心理学が確立されてからはまず類型論的にパーソナリティを記述する研究が盛んに行われた.
Eysenck\cite{eysenck-1963}が四気質を用いたパーソナリティの円環モデルを発表した.
そのモデルの概要を\figref{fig:eysenck}に示す.
また,Kretschmer\cite{kretschmer}は体型によって発症しやすい病気が異なることを示し,体型ごとに気質が異なると考えた.
体型ごとの気質を\tabref{tab:kretschmer}に示す.
他にも,Jung\cite{jung}は精神分析学的な観点から関心・興味が外部に向いている人を外向型,自分自身に向いている人を内向型に分類した.
この外向型,内向型という分類は後のパーソナリティ研究に引き継がれていった.

\begin{figure}[tb]
    \centering
    \includegraphics[width=11truecm]{../image/eysenck.eps}
    \caption[Eysenskの円環モデル]{Eysenskの円環モデル \protect \footnotemark}
    \label{fig:eysenck}
\end{figure}
\footnotetext{出典: Eysenck. \cite{eysenck-1963}(一部修正)}

\begin{table}[tb]
    \centering
    \caption[Kretschmerによる体型と気質の関係]{Kretschmerによる体型と気質の関係 \protect \footnotemark}
    \begin{tabular}{llll} \toprule
        体格 & 気質 & 特徴 & 対応する病理 \\ \midrule
        細長型 & 分裂気質 & 非社交的,無口,神経質 & 統合失調症 \\
        肥満型 & 循環気質 & 社交的,融通がきく,こだわりが弱い & 躁うつ病 \\
        闘士型 & 粘着気質 & 秩序を好む,几帳面,綺麗好き & てんかん \\ \bottomrule
    \end{tabular}
    \label{tab:kretschmer}
\end{table}
\footnotetext{出典: 小塩. \cite{first-personality}(一部修正)}

これまで述べたように,パーソナリティを記述する方法としては古代から20世紀前半まで類型論的なアプローチが主流であった.
類型論はタイプごとの比較が行いやすく,タイプごとに大まかなイメージができるため人々に受け入れられやすいという長所がある.
一方で,少数のタイプで人間の特徴を記述することが難しく,
タイプの特徴を無理に当てはめてしまったり中間的なタイプが記述できないという短所があるため,現在の心理学研究では類型論はあまり扱われていない.
\secref{sec:trait}で説明する特性論は,個人の類似する行動をまとめ,それらの集積としての性格を把握するものである.
そのため,個人の特徴をより正確に捉えることができる.

\subsection{特性論}
\label{sec:trait}
パーソナリティ心理学者であるAllportは類型論では人間の見た目および身体的特徴とパーソナリティが直接関係していることに疑問を持ち,
これらを重視していることがパーソナリティの理解を阻害していると考えた.
Allport\cite{allport}は``character"と``personality"を明確に区別するべきと主張し,
この考えが他の心理学者らに受け入れられることによってpersonalityという用語が定着した.
また,Allportら\cite{allport_odbert}は英語辞典``Webster's New International Dictionary 2nd Edition"に収録されている
約40万語から人間の特徴を表現する約18000語を抽出し,これをさらに4つのカテゴリに分類した.
以降,パーソナリティ特性を語彙を用いて分析するという研究が行われていった.

Cattell\cite{cattell}はAllportらの選んだ単語を因子分析を利用して整理し,意味がよく似た単語と反対の意味の単語をグループにまとめ単語対を作成した.
Cattellは相関係数の高い単語対をまとめていき,最終的に12個の根源特性を導き出した.
Cattellら\cite{16pf}はこれに4つの特性を追加した16個のパーソナリティ特性に基づく16PF(Sixteen Personality Factor Questionnaire)という検査を開発した.

また,Eysenck\cite{mpi}は因子分析を用いて神経症傾向と外向性の2つがパーソナリティの基本的な因子であると考え,
これらを測るためのMPI(Maudsley Personality Inventory)という検査を開発した.
後にEysenckら\cite{epq}は新たに精神病傾向を加えた3つのパーソナリティ因子を測るEPQ(Eysenck Personality Questionnaire)を開発した.

\subsection{ビッグファイブ・5因子モデル}
このように人間のパーソナリティ特性の数は研究者ごとに主張が異なっていたが,
現在,最も多くの研究者が支持しているのがこのように5つのパーソナリティ特性を用いて人間の特徴を記述するものである.
これはビッグファイブ(Big Five)もしくは5因子モデル(Five Factor Model)と呼ばれている.

Norman\cite{norman}は因子分析を行った結果,ビッグファイブ・5因子モデルの前身とも言える
外向性,協調性,誠実性,情緒安定性,文化という5つの因子を見出している.
ビッグファイブはGoldberg\cite{goldberg-1981, goldberg-1990, goldberg-1992}を中心としてAllport以来の語彙研究の流れを汲み,
語彙と因子分析などの統計的処理を用いてパーソナリティの因子を5つに収束させたグループであり,
5因子モデルはMcCraeとCosta\cite{mccrae-1987, mccrae-1992}を中心として複数のパーソナリティ理論をまとめた理論的なアプローチで,
パーソナリティの階層構造を強調するグループである\cite{first-personality}.
ビッグファイブおよび5因子モデルに至るまでの背景はそれぞれ異なっているが,人間のパーソナリティを5つの大きな枠組みで捉えるという点では同じである.
以降,5因子モデルについて触れる必要がある場合を除き,表記はビッグファイブで統一する.
この理論が登場したことにより,今までのパーソナリティ研究を包括的に説明することが可能になった.
また,神経科学や遺伝学の発達により検証可能になった個人間での反応の差異はビッグファイブに当てはめることができる.
これらの点から今日,ビッグファイブはパーソナリティを記述する枠組みとして最も信頼できるものとして利用されている.
前述のようにビッグファイブは海外で生まれた枠組みであるが,日本でも和田\cite{wada}の研究によると日本でもビッグファイブに該当する5因子が確認されている.

ビッグファイブで現在用いられている5つの因子は以下の通りである\cite{daniel}.
\begin{description}
    \item{\textbf{神経症傾向(Neuroticism)}}\mbox{}\\
    神経症傾向(情緒不安定性とも)のスコアが高い人は感情・情緒が不安定で恐怖などに対する警戒心が強く,ストレスを感じやすかったり心配性な傾向がある.
    また,ネガティブな情動は自己の内部へ向けられる傾向があり,スコアが高い人は自身の現状に不満を持つことが多いために人より努力する傾向がある.
    スコアが低い人は情緒的に安定しているが,恐怖などを感じる閾値が非常に高いため自身への脅威を感知できない場合もある.
    神経症傾向のスコアは怒った顔や墓地などのネガティブな画像刺激に対する反応の大きさと相関がある.
    \item{\textbf{外向性(Extraversion)}}\mbox{}\\
    外向性のスコアが高い人は社交的であり,物事に熱中し,社会的地位や自己の目標などの報酬やそれらを達成するための刺激に対する欲求が非常に高い.
    また,報酬を求めるあまりに無謀な行動をする場合があり,大きなリスクを負う可能性があるという側面もある.
    スコアが低い人はよそよそしかったり物静かであり,報酬に対する欲求が低いためリスクを取ってまで報酬を求めることは少ない.
    外向性のスコアは子犬やアイスクリームなどのポジディブな画像刺激に対する反応の大きさと相関がある.
    \item{\textbf{経験への開放性(Openness)}}\mbox{}\\
    経験への開放性のスコアが高い人は好奇心が高く,独創性や創造性が高い.
    一方で異常な信念があったり精神病を患う可能性が高いという傾向がある.
    スコアが低い人は現実的で新しい考え方を受け入れようとしない.
    拡散的思考と経験への開放性の関連が示唆されている.
    \item{\textbf{協調性(Agreeableness)}}\mbox{}\\
    協調性(調和性とも)のスコアが高い人は共感性が高く他者と調和的な社会関係を結ぶことに長けている.
    しかし,自分の利益やステータスを犠牲にしてまで他者への配慮を優先する場合もある.
    スコアが低い人は冷淡で他者に対する配慮をせず,自分の利益を優先し,敵対的な態度を取る場合もある.
    EQ(Emotional Intelligence Quotient)のスコアと協調性のスコアに相関がある.
    \item{\textbf{誠実性(Conscientiousness)}}\mbox{}\\
    誠実性(勤勉性とも)のスコアが高い人は自己を律する能力が高く,欲求やそれによる衝動的行動を抑制することが得意である.
    しかし,自分の計画に固執しすぎて融通が利かないという側面がある.
    スコアが低い人は集中力がなく,衝動的であり,物事を先延ばしにする傾向がある.
    ギャンブルや薬物への依存性の高さと誠実性のスコアの低さに相関がある.
\end{description}

ビッグファイブは\figref{fig:big5}のように個人が各因子をどの程度持っているかを測ることでその個人のパーソナリティを記述する.
このように特性論的なアプローチは類型論的なアプローチよりも個人の特徴を的確に記述することができる.
一方でビッグファイブは各因子が独立であるという前提に立っているが,CostaとMcCrae\cite{neo-pi-r}は神経症傾向と誠実性に負の相関,
外向性と開放性に正の相関があると述べており,すべての因子が完全に独立しているわけではないといえる\cite{eysenck-handbook}.
このようにビッグファイブの理論にもいくつか問題点が指摘されている.
ビッグファイブの測定法には様々なものが考案されているが,\secref{sec:mesure}ではその一部について述べる.

\begin{figure}[tb]
    \centering
    \includegraphics[width=10truecm]{../image/big5.eps}
    \caption{ビッグファイブによるパーソナリティの表現}
    \label{fig:big5}
\end{figure}

\subsection{ビッグファイブの測定法}
\label{sec:mesure}
ビッグファイブを測定するためにGoughら\cite{acl}のACL(Adjective Check List)が用いられることがある.
ACLは性格を表すような形容詞で構成されており,文章形式の質問項目よりも構造が安定している.
ACLはビッグファイブを測定するために開発されたものではないが,Piedmontら\cite{big5-acl}はACLとビッグファイブとの関連性を示唆している.
日本語版のACLは柏木ら\cite{acl-ja}によって作成されている.

また,5因子モデルの各因子を測定する手法としてCostaとMcCrae\cite{neo-pi-r}が開発したNEO-PI-R(Revised NEO Personality Inventory)がある.
NEO-PI-Rでは前述の5因子を上位次元とし,各因子にはそれぞれ6つの下位次元が存在している.
これら30個の下位次元に対して質問項目が用意されており,これにより各次元の値を測定する.
日本語版のNEO-PI-Rは下仲ら\cite{neo-pi-r-ja}によって作成されている.
\tabref{tab:neo-pi-r}に各因子の下位次元を示す.

上述の測定法は質問項目が多く,被験者の負担や時間的制約などから使用が難しい場合があるため,
Goslingら\cite{tipi}による各因子2項目ずつの計10項目でビッグファイブを計測するTIPI(Ten Item Personality Inventory),
小塩ら\cite{tipi-j}によるTIPIの日本語版であるTIPI-Jなど少ない項目でビッグファイブを測定する方法も多く考案された.
しかし,このような短い項目では因子が十分に測定できない可能性がある.

また,Nettle\cite{daniel}はIPIP(International Personality Item Pool)尺度を利用することを有効性・利用の容易さの観点から勧めている.
同時に,IPIP尺度の中でもできるだけ長い質問項目を利用するべきと述べている.

\begin{table}[tb]
    \centering
    \caption[5因子の下位次元]{5因子の下位次元 \protect \footnotemark}
    \begin{tabular}{ll} \toprule
        因子 & 下位次元 \\ \midrule
        神経症傾向 & 不安,敵意,抑うつ,自意識,衝動性,傷つきやすさ \\
        外向性 & 温かさ,群居性,断行性,活動性,刺激希求性,良い感情 \\
        開放性 & 空想,審美性,感情,行為,アイデア,価値 \\
        協調性 & 信頼,実直さ,利他性,応諾,慎み深さ,優しさ \\
        誠実性 & コンピテンス,秩序,良心性,達成追求,自己鍛錬,慎重さ \\ \bottomrule
    \end{tabular}
    \label{tab:neo-pi-r}
\end{table}
\footnotetext{出典: 無藤ら. \cite{psychology} (一部修正)}

\section{TKIを用いた人間とエージェントの交渉に関する研究}
\secref{sec:introduction}でも述べたが,Koleyら\cite{tki-ha}はTKIを用いたエージェントを使って人間との交渉を行った.
自動交渉でも藤田\cite{tki-aa}がTKIを使用して譲歩速度を変更するエージェントを提案しており,Kolerらはそれを人間とエージェントの交渉に応用した.
藤田は過去の交渉セッションでのエージェントの提案と現在のエージェントの提案を比較することで相手のモードを推定していた.
しかし,Koleyらは過去の交渉のデータを使用しない.すなわち,現在の提案と同一の交渉内での過去の提案を比較することによって相手のモードを推定している.
これにより,Koleyらのエージェントはあらゆる相手に適応できる柔軟性を持つことができると述べている.

藤田およびKoleyらは効用の平均と分散で相手の協調性,積極性を測り相手のモードを推定した.
\tabref{tab:tki-metrics}に協調性,積極性の推定方法を示す.

\begin{table}[tb]
    \centering
    \caption[協調性および積極性の推定]{協調性および積極性の推定 \protect \footnotemark}
    \begin{tabular}{llll} \toprule
        条件 & 協調性 & 条件 & 積極性 \\ \midrule
        $U_{eH}(bid_t)>\mu_e$ & 非協調的 & $\sigma^2_e(t)>\sigma^2_{eh}$ & 消極的 \\
        $U_{eH}(bid_t)=\mu_e$ & 中立 & $\sigma^2_e(t)=\sigma^2_{eh}$ & 中立 \\
        $U_{eH}(bid_t)<\mu_e$ & 協調的 & $\sigma^2_e(t)<\sigma^2_{eh}$ & 積極的\\ \bottomrule
    \end{tabular}
    \label{tab:tki-metrics}
\end{table}
\footnotetext{出典: Koley et al. \cite{tki-ha},藤田\cite{tki-aa} (一部修正)}

ラウンド$t$における相手が送信した提案を$bid_t$,$bid_t$によって人間が得られる推定効用を$U_{eH}(bid_t)$,相手が送信した過去の提案で人間が得られる推定効用の平均を$\mu_e$,
$bid_t$によって人間が得られる推定効用の分散を$\sigma^2_e(t)$,相手が送信した過去の提案で人間が得られる推定効用の分散を$\sigma^2_{eh}$とする.
$U_{eH}(bid_t)$が$\mu_e$より低い場合,相手は譲歩しており,協調的だとみなす.
一方,$U_{eH}(bid_t)$が$\mu_e$より高い場合,相手は譲歩していないため相手は非協調的だとみなす.
また,$\sigma^2_e(t)$が$\sigma^2_{eh}$より低い場合,相手は同じ提案を多く送信しており,積極的だとみなす.
一方,$\sigma^2_e(t)$が$\sigma^2_{eh}$より高い場合,相手の提案は分散しているため相手は消極的だとみなす.

Koleyらのエージェントは\equref{eq:target},\equref{eq:delta},\equref{eq:gamma}で表される目標効用$target(t,\alpha)$が得られるような提案を行う.
\begin{align}
    target(t,\alpha) &= \gamma_{min} + \Delta \cdot \Gamma(t,\alpha)  \label{eq:target}\\
    \Delta &= \gamma_{max} - \gamma_{min} \label{eq:delta} \\
    \Gamma(t,\alpha) &= 1 - U_A(bid_{t - 1}) \cdot \left( \frac{t}{n} \right)^{\frac{1}{\alpha}} \label{eq:gamma}
\end{align}

\begin{itembox}[l]{各変数の定義}
    $t$ : 現在のラウンド数 \\
    $\alpha \in [0,1]$ : 譲歩速度 \\
    $\gamma_{min}$ : エージェントの目標効用の最低値 \\
    $\gamma_{max}$ : エージェントの目標効用の最高値 \\
    $U_A(bid_{t - 1})$ :$bid_{t - 1}$によってエージェントが得られる効用 \\
    $n$ : 総ラウンド数
\end{itembox}

\begin{figure}[tb]
    \centering
    \includegraphics[width=7truecm]{../image/target_utility.eps}
    \caption[Koleyらの目標効用関数]{Koleyらの目標効用関数 \protect \footnotemark}
    \label{fig:koley_utility}
\end{figure}
\footnotetext{出典: Koley et al. \cite{tki-ha}}

エージェントの目標効用は探索する効用空間の大きさを表す$\Delta$と,ラウンド数とラウンド$t - 1$における提案に依存する$\Gamma(t,\alpha)$によって決定する.
すなわち,交渉初期のラウンド数が小さい場合やラウンド$t - 1$における提案でエージェントが得られる効用が低い場合は相手の譲歩を引き出すために目標効用が高くなる.
$\alpha$は協調性と積極性を\tabref{tab:tki-metrics}の方法で測定し,相手のモードが妥協モードか順応モードであった場合に0.1ずつ増加していく.
したがって,相手が譲歩した回数が多いほど$\alpha$が大きくなり,\figref{fig:koley_utility}のように譲歩速度が速くなる.

Koleyらは上述のようなエージェントを作成し,人間と交渉する実験を行った.
\figref{fig:koley_time}のように24人の被験者のうち23人が時間内に合意に至り,その平均値は約392秒であった.
また,\figref{fig:koley_result}のように合意案のほとんどがパレート最適解近辺であり,さらにKalai-Smorodinsky解\cite{kalai}である(0.7, 0.7)の近辺が最も多かった.
Koleyらは\equref{eq:rms},\equref{eq:efficient}によって合意に至った場合の交渉結果の有効性を計算しており,約97.7\%という値となった.
\begin{align}
    RMS &= \sqrt{\frac{\sum_{i=1}^N euclideanDist(result_i, paretoFront)^2}{N}} \label{eq:rms} \\
    \mathit{EfficacyScore} &= \left( 1 - \frac{RMS}{\sqrt{0.7^2 + 0.7^2}} \right) \label{eq:efficient}
\end{align}

\begin{figure}[bt]
    \centering
    \includegraphics[width=15truecm]{../image/koley_time.eps}
    \caption[Koleyらの研究における合意到達までの所要時間]{Koleyらの研究における合意到達までの所要時間 \protect \footnotemark}
    \label{fig:koley_time}
\end{figure}
\footnotetext{出典: Koley et al. \cite{tki-ha}}

\clearpage

\begin{figure}[!tb]
    \centering
    \includegraphics[width=13truecm]{../image/koley_result.eps}
    \caption[Koleyらの交渉結果]{Koleyらの交渉結果 \protect \footnotemark}
    \label{fig:koley_result}
\end{figure}
\footnotetext{出典: Koley et al. \cite{tki-ha}}

\bibliography{reference}
% 付録用
%\chapter*{付録}


\ifthmaster
  \externals
\begin{externalsenum}{H}
\item \underline{松下昌悟},藤田桂英.
    心理的効果を用いた人間とエージェントの繰り返し交渉戦略.
    電子情報通信学会 人工知能と知識処理研究会(AI), July 2019.
\end{externalsenum}

\fi

\end{document}