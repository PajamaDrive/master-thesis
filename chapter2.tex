\documentclass[a4paper,11pt]{jreport}
\usepackage[master]{tuats}
\debugtrue

% PDF用メタデータ
% コメントアウトすれば著者名やタイトルをPDFに埋められるが,問題が発生しがち
% \AtBeginDvi{
%     \special{pdf:tounicode EUC-UCS2}
%     \special{pdf:docinfo <<
%         /Author (安井貴規 (Takaki YASUI))
%         /Title  (藤田桂英研究室栄光のLaTeXフォーマット (A LaTeX Style Files for Fujita Katsuhide Labratory))
%     >>}
% }

%タイトル
\title{人間とエージェントの交渉における \\ プロファイリングを用いた交渉戦略}
\etitle{Strategy using profiling for human-agent negotiation}

%名前
\author{松下昌悟}
\eauthor{Shogo Matsushita}

%入学年度
\enteryear{2019}
%卒業年度
\graduateyear{2020}

%学籍番号
\studentnumber{19646029}

%提出日
\date{2021年1月29日}
\begin{document}

\chapter{関連研究}
\section{自動交渉エージェント競技会(ANAC)}

自動交渉エージェント競技会(Automated Negotiation Agent Competition)は各自が作成したエージェント同士で交渉を行い
個人獲得効用や社会的余剰などを競う競技会である\cite{anac2010-2017,anac2018,anac2019,anac2020}.
ANACは毎年開催され,2010年(第1回)から2016年(第7回)まではInternational Conference on Autonomous Agents and Multiagent Systems(AAMAS)と共催で行われた.
2017年からはInternational Joint Conference on Artificial Intelligence(IJCAI)と共催で行われ,
2020年(第11回)もIJCAIと共催で行われる予定である.
ANACは以下の4点を目的としている.

\begin{itemize}
    \item 未知の相手に対し,様々な状況で巧みに交渉できる実用的な交渉エージェントの設計を促進する
    \item 様々な交渉戦略を客観的に評価する指標を提供する
    \item 様々な学習手法,適応戦略,交渉相手のモデル構築を探求する
    \item 最先端の交渉エージェントと交渉シナリオを収集し,研究コミュニティに提供する
\end{itemize}

ANACでは2016年までは自動交渉のプラットフォームであるGeniusを用いて単一のリーグが開催されていた.
2017年からは複数のリーグが開催され,2020年は以下の5つのリーグが開催される予定である.

\begin{itemize}
    \item Automated Negotiation League(エージェント同士での交渉)
    \item Human-Agent League(人間とエージェントの交渉)
    \item Supply Chain Management League(サプライチェーンにおける交渉)
    \item Werewolf Game League(人狼ゲームにおける交渉)
    \item HUMAINE(対話形式の人間と複数エージェントとの交渉)
\end{itemize}

\section{IAGO}
IAGO(Interactive Arbitration Guide Online)はANACのHuman-Agent Negotiationリーグで使用される自動交渉のプラットフォームである\cite{iago}.
IAGOのインタフェースを\figref{fig:iago}に示す.

\begin{figure}[htb]
    \centering
    \includegraphics[width=15truecm]{image/IAGO.eps}
    \caption{IAGOのインタフェース}
    \label{fig:iago}
\end{figure}

IAGOでは,人間と交渉することが可能なエージェントおよび人間とエージェントの交渉に用いるドメインを作成することが可能である.
IAGOの主な特徴として以下の3点が挙げられる\cite{pinocchio}.

\begin{enumerate}
    \item 作成したエージェントはWEBサーバ上で動作し,ユーザはソフトウェアのインストールなどを行うことなくWEBブラウザ上で簡単にエージェントと交渉を行うことができる
    \item エージェントおよび交渉ドメインの設計・開発を独立して行えるようにAPIが展開されている
    \item 人間同士の交渉で用いられる行動・チャネルが使用可能である
\end{enumerate}

交渉エージェントの開発を行うプラットフォームとしてGenius\cite{genius}やNegMAS\cite{negmas}などがあるが,
これらのプラットフォームで開発されるエージェントはエージェント同士での交渉を前提としているため,
感情やメッセージの送受信など交渉で人間が行う行動を取り扱うことができない.
しかし,人間同士の交渉と同様に人間とエージェントの交渉でも怒りや喜びなどの感情が交渉結果に影響をおよぼすことが明らかになっている\cite{emotion}.

一方,IAGOは人間と交渉できるエージェントを開発し,研究・シミュレーションを行うことができるように設計されている.
そのため,IAGOで行われる交渉において人間およびエージェントは他のプラットフォームではサポートされていない様々な行動を行うことが可能である.
交渉者は部分的な提案を行うことが可能で,さらに提案だけでなく感情の表出やメッセージの送信,お互いの選好に関する情報を送受信することができる.
以上のような人間同士の交渉で用いられる諸要素をユーザおよびエージェントが使用することができるため,
実際に行われる交渉と近い状況で人間との交渉をシミュレーションすることができる.
このように,人間と交渉可能なエージェントを開発することができるため,IAGOを用いて実験を行う研究が増えつつある.
また,最新版のIAGOでは留保価格についての情報交換や見返りの要求や約束が行えるようになっており,より現実に即した交渉を行うことが可能になった.
最新版のIAGOを用いて見返りを求めるエージェントの方が高い効用を得るという研究結果が出ている\cite{favor}.

\section{TKI}
\label{sec:tki}
TKIは対立した状況における個人の行動を評価・分析するための指標である\cite{tki}.
TKIでは積極性(assertiveness)と協調性(cooperativeness)の2つの軸を用いる.
積極性は自分自身の利益や要求を満たそうとする程度を表し,協調性は相手の利益や要求を満たそうとする程度を表す.
TKIでは積極性と協調性の値によって以下の5つのモードに分類される.
\begin{description}
    \item{\textbf{競争(Competing)}}\mbox{}\\
    積極性が高く,協調性が低い人は競争モードに分類される.
    競争モードは衝突が起こった際,相手を犠牲にして自分の利益を追求する.
    また,自分の権利のために立ち上がり,自分が正しいと信じた信念を守り,相手に勝利するために行動する.
    \item{\textbf{共存(Collaborating)}}\mbox{}\\
    積極性が高く,協調性も高い人は共存モードに分類される.
    共存モードは相手と一緒に互いの要求を満たすような解決策を探る.
    また,両者の根本的な要求を把握し,両者の要求を満たすような代替案を探るために問題を深掘りする.
    \item{\textbf{回避(Avoiding)}}\mbox{}\\
    積極性が低く,協調性も低い人は回避モードに分類される.
    回避モードは自分自身または相手の利益を追求することはない.
    また,自ら問題に対処することはなく,問題のすり替え,問題への対処を延期,問題からの撤退などの形を取る可能性がある.
    \item{\textbf{順応(Accommodating)}}\mbox{}\\
    積極性が低く,協調性が高い人は順応モードに分類される.
    順応モードは親身であり,自分の利益を無視してでも相手を満足させる.
    このように順応モードは自己犠牲の精神があり,相手の命令や脅迫に従うことがある.
    \item{\textbf{妥協(Compromising)}}\mbox{}\\
    積極性と協調性が中庸な人は妥協モードに分類される.
    妥協する際の目標は両者が部分的に満足でき,互いに受け入れることができるような内容のものである.
    妥協の度合いとしては競争モードよりは譲歩し順応モードよりは譲歩しない.
    また,回避モードよりは問題に対処しようとするが共存モードのように問題を深掘りするわけではない.
    このように妥協モードは全てのモードの中間の行動をとる.
\end{description}

\begin{figure}[htb]
    \centering
    \includegraphics[width=11truecm]{image/tki.eps}
    \caption{TKIによるパーソナリティの表現}
    \label{fig:tki}
\end{figure}

\section{ビッグファイブ・5因子モデル}
ビッグファイブ(Big Five)もしくは5因子モデル(Five Factor Model)は人間のパーソナリティを5つの因子で表現するものである.
まず,ビッグファイブ・5因子モデルが提唱されるまでの経緯について簡単に述べる\cite{psychology, first-personality, daniel}.

\secref{sec:tki}で述べたTKIのように人間のパーソナリティをいくつかの大きなグループに分類し,
グループごとに典型的な性格や特徴などを記述する手法を類型論と呼ぶ.
類型論的なパーソナリティの記述は古代から行われている.
人間のパーソナリティについて記述された現存する世界最古の書物は,紀元前に哲学者であり植物学者でもあるテオプラストスによって書かれた“人さまざま"であるとされている.
この書物の中ではへそまがり,お節介,傲慢などの30種類のパーソナリティを定義し,その特徴や具体的な人物像について記述されている.
類型論的な記述は古代から現代まで長きに渡り使用されており人々にとって理解しやすいものが多い一方,中間的なパーソナリティを記述しづらいという特徴がある.

パーソナリティ心理学者であるAllportは類型論では人間の見た目および身体的特徴とパーソナリティが直接関係していることに疑問を持ち,
これらを重視していることがパーソナリティの理解を阻害していると考えた.
Allportは“character"と“personality"を明確に区別するべきと主張し\cite{allport},
この考えが他の心理学者らに受け入れられることによってpersonalityという用語が定着した.
また,Allportらは英語辞典“Webster's New International Dictionary 2nd Edition"に収録されている
約40万語から人間の特徴を表現する約18000語を抽出し,これをさらに4つのカテゴリに分類した\cite{allport_odbert}.
以降,パーソナリティ特性を語彙を用いて分析するという研究が行われていった.

CattellはAllportらの選んだ単語を因子分析を利用して整理し,意味がよく似た単語と反対の意味の単語をグループにまとめ単語対を作成した\cite{cattell}.
Cattellは相関係数の高い単語対をまとめていき,最終的に12個の根源特性を導き出した.
Cattellらはこれに4つの特性を追加した16個のパーソナリティ特性に基づく16PF(Sixteen Personality Factor Questionnaire)という検査を開発した\cite{16pf}.

また,Eysenckは因子分析を用いて神経症傾向と外向性の2つがパーソナリティの基本的な因子であると考え,
これらを測るためのMPI(Maudsley Personality Inventory)という検査を開発した\cite{mpi}.
後にEysenckらは新たに精神病傾向を加えた3つのパーソナリティ因子を測るEPQ(Eysenck Personality Questionnaire)を開発した\cite{epq}.

このように人間のパーソナリティ特性の数は研究者ごとに異なっていたが,
現在,最も多くの研究者が支持しているのが5つのパーソナリティ特性を用いて人間の特徴を記述するものである.
これはビッグファイブもしくは5因子モデルと呼ばれている.
ビッグファイブはGoldberg\cite{goldberg-1981, goldberg-1990, goldberg-1992}を中心としてAllport以来の語彙研究の流れを汲み,
語彙と因子分析などの統計的処理を用いてパーソナリティの因子を5つに収束させたグループであり,
5因子モデルはMcCraeとCosta\cite{mccrae-1987, mccrae-1992}を中心として複数のパーソナリティ理論をまとめた理論的なアプローチで,
パーソナリティの階層構造を強調するグループである\cite{first-personality, tipij}.
ビッグファイブおよび5因子モデルに至るまでの背景はそれぞれ異なっているが,人間を5つの大きな枠組みで捉えるという点では同じである.
以降,表記はビッグファイブで統一する.
この理論が登場したことにより,今までのパーソナリティ研究を包括的に説明することが可能になった.
また,神経科学や遺伝学の発達により検証可能になった個人間での反応の差異はビッグファイブに当てはめることができる.
これらの点から今日,ビッグファイブはパーソナリティを記述する枠組みとして最も信頼できるものとして利用されている.

ビッグファイブで用いられている5つの因子は以下の通りである.
\begin{description}
    \item{\textbf{神経症傾向(Neuroticism)}}\mbox{}\\
    神経症傾向が高い人は感情・情緒が不安定で恐怖などに対する警戒心が強くストレスを感じやすい.
    また,現状に不満を持つために人より努力する傾向がある.
    \item{\textbf{外向性(Extraversion)}}\mbox{}\\
    外向性が高い人は活動的で,報酬や刺激に対する欲求が非常に高い.
    報酬を求めるあまりにリスクを負いやすいという側面もある.
    \item{\textbf{経験への開放性(Openness)}}\mbox{}\\
    経験への開放性が高い人は好奇心が高く,創造性が高い.
    一方で異常な信念があったり精神病を患う可能性が高い傾向がある.
    \item{\textbf{協調性(Agreeableness)}}\mbox{}\\
    協調性が高い人は共感性が高く他者と調和的な社会関係を結ぶことに長けている.
    しかし,自分の利益やステータスを犠牲にしてまで他者への配慮を優先する場合もある.
    \item{\textbf{誠実性(Conscientiousness)}}\mbox{}\\
    誠実性が高い人は自己を律する能力が高く,欲求やそれによる衝動的行動を抑制することが得意である.
    しかし,自分の計画に固執しすぎて融通が利かないという面がある.
\end{description}

ビッグファイブはこれら5つの因子がどのような数値であるか測ることによって人間の性格・特性を表すものであり,このような手法を特性論と呼ぶ.
特性論はTKIのような類型論と比較して中間的なパーソナリティを記述しやすいという特徴がある.

\begin{figure}[htb]
    \centering
    \includegraphics[width=10truecm]{image/big5.eps}
    \caption{ビッグファイブによるパーソナリティの表現}
    \label{fig:big5}
\end{figure}

\section{TKIを用いた人間とエージェントの交渉に関する研究}
TKIを用いたエージェントを使って人間との交渉を行った研究がある\cite{tki-ha}.
この研究では相手に提案を行う際の譲歩関数のパラメータを決定するためにTKIを用いている.相手の提案における相手の効用の平均値$\mu$で協調性を,効用の分散$\sigma^2$で積極性を計算する.前回の平均値よりも$\mu$が小さければ協調的,前回の分散よりも$\sigma^2$が小さければ積極的とみなす.
上述のようにして相手の積極性と協調性を測り,相手が妥協もしくは順応モードであるときに譲歩する.
このようなエージェントを作成し,人間との実験を行った.24人の被験者と交渉を行い,うち23人が時間内に合意に至った.また,合意案のほとんどがパレートフロント近辺であった.

\bibliography{reference}
% 付録用
%\chapter*{付録}


\ifthmaster
  \externals
\begin{externalsenum}{H}
\item \underline{松下昌悟},藤田桂英.
    心理的効果を用いた人間とエージェントの繰り返し交渉戦略.
    電子情報通信学会 人工知能と知識処理研究会(AI), July 2019.
\end{externalsenum}

\fi

\end{document}