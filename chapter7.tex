\documentclass[a4paper,11pt]{jreport}
\usepackage[master]{tuats}
\debugtrue

% PDF用メタデータ
% コメントアウトすれば著者名やタイトルをPDFに埋められるが,問題が発生しがち
% \AtBeginDvi{
%     \special{pdf:tounicode EUC-UCS2}
%     \special{pdf:docinfo <<
%         /Author (安井貴規 (Takaki YASUI))
%         /Title  (藤田桂英研究室栄光のLaTeXフォーマット (A LaTeX Style Files for Fujita Katsuhide Labratory))
%     >>}
% }

%タイトル
\title{人間とエージェントの交渉における \\ プロファイリングを用いた交渉戦略}
\etitle{Strategy using profiling for human-agent negotiation}

%名前
\author{松下昌悟}
\eauthor{Shogo Matsushita}

%入学年度
\enteryear{2019}
%卒業年度
\graduateyear{2020}

%学籍番号
\studentnumber{19646029}

%提出日
\date{2021年1月29日}
\begin{document}

\chapter{おわりに}
\section{まとめ}
本稿では,TKI,ビッグファイブを用いて相手の特性を推定することで人間と円滑に交渉を行い,信頼関係を築きながら効用を高められるようなエージェントを提案した.
提案内容や相手の行動からTKIのモード,ビッグファイブの各因子の値を求めるために使用する交渉の要素を定義し,これらの定式化を行なった.
また,部分的な提案を許容し,それに応じて既存研究で使用されていた目標効用関数の拡張,考慮されていなかったモードに対する戦略の提案を行い,
より汎用的なエージェントおよび戦略を提案した.
ビッグファイブの各因子を実際に測定できるか確認し,パラメータを調整するために予備実験を行った.

また,予備実験で決定したパラメータを用いてTKIとビッグファイブを併用するエージェントの有効性とビッグファイブ測定の精度を評価するための実験を行った.
実験の結果,交渉中にビッグファイブを測定する動的測定エージェントは高い社会的余剰を実現しつつ,相手に良い印象を与えており,なおかつ現実的な提案を行う
ことができたと評価された.
また,事前に測定したビッグファイブの値を使用した静的測定エージェントは短い交渉時間で高い効用を獲得することができ,協調的な印象であるという評価であった.
これらより,動的測定,静的測定エージェントはどちらも異なる点で既存手法より優れていることが示された.

\section{今後の課題}
以下に今後の課題,および展望を示す.

\begin{description}
    \item{\textbf{TKIの測定に関する課題}}\mbox{}\\
    本稿ではTKIのモードを測定するために提案の効用の平均,効用の標準偏差のみを用いていた.
    既存研究では部分的な提案を許容していなかったため,これらによって協調性,積極性を測定することで効果的な提案を行うことができた.
    しかし,部分的な提案を許容した場合,効用の平均と標準偏差のみでこれらの測定を行うことは困難である,
    そのため,TKIの協調性,積極性の測定方法についても議論が必要である.
    \item{\textbf{ビッグファイブの測定に関する課題}}\mbox{}\\
    本稿で定式化したビッグファイブの各因子の測定方法はすべて現在のIAGOのバージョンで利用できる行動に完全に依存している.
    そのため,他のプラットフォームでは同様な測定を行うことができない.
    また,人間同士の交渉では相手の仕草,フレーズなどIAGOでは提供されていない要素から相手の特性を判断することが可能であり,
    それらの要素をエージェントも考慮するべきである.
    このように,本稿で定式化した測定方法では相手の特性を完全に把握することは困難である.
    \item{\textbf{被験者数を増やして実験を再試行}}\mbox{}\\
    予備実験,評価実験ともに被験者数が少なく,十分なサンプルが取れなかった.
    これにより,サンプルの偏りによって有意差が出なかった,もしくは偏った結果になっている可能性がある.
    したがって,被験者数を増やしてパラメータ調整や評価実験を行う必要がある.
\end{description}

\bibliography{reference}
% 付録用
%\chapter*{付録}


\ifthmaster
  \externals
\begin{externalsenum}{H}
\item \underline{松下昌悟},藤田桂英.
    心理的効果を用いた人間とエージェントの繰り返し交渉戦略.
    電子情報通信学会 人工知能と知識処理研究会(AI), July 2019.
\end{externalsenum}

\fi

\end{document}