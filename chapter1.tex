\documentclass[a4paper,11pt]{jreport}
\usepackage[master]{tuats}
\usepackage{ascmac}
\usepackage{mathtools}
\usepackage{booktabs}
\debugtrue

% PDF用メタデータ
% コメントアウトすれば著者名やタイトルをPDFに埋められるが,問題が発生しがち
% \AtBeginDvi{
%     \special{pdf:tounicode EUC-UCS2}
%     \special{pdf:docinfo <<
%         /Author (安井貴規 (Takaki YASUI))
%         /Title  (藤田桂英研究室栄光のLaTeXフォーマット (A LaTeX Style Files for Fujita Katsuhide Labratory))
%     >>}
% }

%タイトル
\title{人間とエージェントの交渉における \\ プロファイリングを用いた交渉戦略}
\etitle{Strategy using profiling for human-agent negotiation}

%名前
\author{松下昌悟}
\eauthor{Shogo Matsushita}

%入学年度
\enteryear{2019}
%卒業年度
\graduateyear{2021}

%学籍番号
\studentnumber{19646029}

%提出日
\date{2021年1月X日}
\begin{document}

\chapter{はじめに}
\section{研究背景}
交渉は要求や意見の異なる他者と共同で意思決定を行うプロセスであり,社会生活を営む上で必要不可欠な工程である.
交渉スキルは友人への頼みごと等の日常的な小規模な問題だけでなく企業の提携,国家間の取引等の大規模な問題を解決する際に必要である.
特に,組織的な団体に属する場合は複数人で意思決定を行う場合が多く,交渉を行う場面が多い\cite{negSurvey}.
しかし,交渉スキルなどの対人能力は大学を卒業した学生であっても不十分であり\cite{graduate},交渉スキルの不足により不利益を被ることも多い.
交渉スキルを高めるには専門家による指導を受け,体験学習で実践を行う必要があり,受講の費用も高額であるため習得コストが非常に高い.
このように交渉スキルを高めるためのコストは非常に高いが,教育ツールとして交渉エージェントを包含したソフトウェアを用いることで習得にかかるコストを劇的に削減することができる.
また,専門家による指導は指導者と受講者の時間的制約もあるが,ソフトウェアによる指導の場合は時間的制約も削減することが可能である.
交渉はビジネス,衝突解消,AIなど複数の分野で研究されているが,前述のように交渉スキルを教育するためのツールなどへ応用が可能であるため,
人間とエージェントとの交渉への関心が高まっている\cite{vr}.

人間とエージェントが交渉を行う際は感情など人間に特有の心理的要素が交渉結果に影響を与えるが,人間とエージェントの交渉ではこれらの影響を考慮していないモデルが多かった.
その要因として,エージェント同士の交渉に用いるエージェントを作成するプラットフォームとしてGenius\cite{genius}などがある一方で,人間と交渉可能なエージェントを作成するために最適なプラットフォームがないという問題点があった.
人間と交渉を行うことができるエージェントを作成するためのプラットフォームであるIAGO\cite{iago}は感情やメッセージの送信を行うためのチャネルがあらかじめ用意されており,IAGOが登場してからは,心理的効果を反映したエージェントに関する研究が増えつつある.


その中で,人間をいくつかのタイプに分け,タイプによって戦略を変えるエージェントが良い交渉結果を得ている\cite{tki-ha}.
しかし,この研究で用いられているエージェントは現在の提案と過去の提案を比較することのみで譲歩するかどうかを決定しており,極端な提案を成されたときに対応が困難,相手の特性を俯瞰的に見極めることが困難であるといった問題点がある.

現実で行わている人間同士の交渉では,交渉を円滑に行い信頼関係を築くために相手の表情や仕草など提案内容以外の要素を考慮しながら交渉が行われている.それにも関わらず,人間とエージェントの交渉に関する研究では相手の提案内容のみによって戦略を変更する場合がほとんどであるのが現状である.

\section{本研究の目的}
本研究では,人間と円滑に交渉を行い,信頼関係を築きながら効用を高められるようなエージェントを提案することを目的とする.
また,交渉中の相手の行動や提案内容によってビッグファイブの各因子を計測することを試みる.
具体的には,TKI・ビッグファイブといった人間の特性を測るための手法を使用し,相手の特性に合わせた戦略をとることで,円滑な交渉を実現する.
また,提案手法および既存研究のエージェントと人間が交渉を行う実験を実施し評価を行うことで,提案手法の有効性を示すことを目的とする.

\section{本論文の構成}
以下に本論文の構成を述べる.第2章では,関連研究として,自動交渉エージェント競技会,IAGO,プロファイリング手法についての概要,
TKIを利用した人間とエージェントの交渉についての研究について述べる.
第3章では,本研究で取り扱う交渉問題である複数論点交渉問題について述べる.
第4章では,提案手法としてプロファイリング手法を用いた交渉戦略について述べる.
第5章では,提案手法で用いるパラメータを調整するために行った予備実験の概要・結果について述べる.
第6章では,予備実験で決定したパラメータを用いて被験者に対して行なった評価実験の概要・結果について述べる.
最後に,第7章では本研究のまとめと今後の課題を示す.

\bibliography{reference}
% 付録用
%\chapter*{付録}


\ifthmaster
  \externals
\begin{externalsenum}{H}
\item \underline{松下昌悟},藤田桂英.
    心理的効果を用いた人間とエージェントの繰り返し交渉戦略.
    電子情報通信学会 人工知能と知識処理研究会(AI), July 2019.
\end{externalsenum}

\fi

\end{document}