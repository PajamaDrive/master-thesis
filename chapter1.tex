\documentclass[a4paper,11pt]{jreport}
\usepackage[master]{tuats}
\usepackage{ascmac}
\usepackage{mathtools}
\usepackage{booktabs}
\debugtrue

% PDF用メタデータ
% コメントアウトすれば著者名やタイトルをPDFに埋められるが,問題が発生しがち
% \AtBeginDvi{
%     \special{pdf:tounicode EUC-UCS2}
%     \special{pdf:docinfo <<
%         /Author (安井貴規 (Takaki YASUI))
%         /Title  (藤田桂英研究室栄光のLaTeXフォーマット (A LaTeX Style Files for Fujita Katsuhide Labratory))
%     >>}
% }

%タイトル
\title{人間とエージェントの交渉における \\ プロファイリングを用いた交渉戦略}
\etitle{Strategy using profiling for human-agent negotiation}

%名前
\author{松下昌悟}
\eauthor{Shogo Matsushita}

%入学年度
\enteryear{2019}
%卒業年度
\graduateyear{2021}

%学籍番号
\studentnumber{19646029}

%提出日
\date{2021年1月X日}
\begin{document}

\chapter{はじめに}
\section{研究背景}
\label{sec:introduction}
Gratchら\cite{han-challenge}は次のように述べている.
``Negotiation is an indispensable skill for any social creature"(交渉は社会的生物に必要不可欠なスキルである[筆者訳]).
交渉は要求や意見の異なる他者と共同で意思決定を行うプロセスであり,社会生活を営む上で必要不可欠な工程である.
交渉スキルは友人への頼みごと等の日常的で小規模な問題だけでなく組織内での意思決定,企業の提携,国家間の取引等の大規模な問題を解決する際に必要である.
特に,組織的な団体に属する場合は複数人で意思決定を行う場合が多く,交渉が行われる場面が多い\cite{negSurvey}.
交渉はビジネス,衝突解消,AIなど複数の分野で研究されているが,人間と交渉ができるエージェントの開発はこれらの複数の分野での重要な課題であり,
人間とエージェントとの交渉への関心が高まっている\cite{han-challenge}.

交渉スキルなどの対人能力は大学を卒業した学生であっても不十分であり\cite{graduate},交渉スキルの不足により不利益を被ることも多い.
交渉スキルを高めるために企業は従業員に対して専門家による指導を受けさせ,体験学習を行う必要があるなど非常に高いコストをかけている\cite{nego-teach}.
交渉スキルを向上させるためのコストは非常に高いが,教育ツールとして交渉できるエージェントを包含したソフトウェアを用いることで習得にかかるコストを劇的に削減することができる.
また,対人による交渉スキルの訓練は指導者と受講者の時間的制約もあるが,ソフトウェアによる訓練の場合は時間的制約もある程度排除できる.
Broekensら\cite{vr-training}やCoreら\cite{vr-simulation}はバーチャルエージェントを用いた訓練が交渉スキルの向上に役立つことを示している.

AIが普及しつつある昨今,交渉を人間を介さずにエージェント同士で自律的に行わせる自動交渉が意思決定を行う手段として注目されている\cite{automated-negotiation}.
自動交渉は合理的判断を下せるエージェント同士が交渉するため,心理的効果などによって非合理的な判断を下すことはない.
実際,自動交渉に用いるエージェントの開発を行うプラットフォームとしてGenius\cite{genius}やNegMAS\cite{negmas}などがあるが,
これらのプラットフォームには感情などを表出する機能はない.

一方で,近年人間同士の交渉への感情の影響に関心が高まっている\cite{emotion-effect}.
人間同士の交渉では怒りを表出した場合は相手が譲歩しやすく,喜びを表出した場合は相手があまり譲歩しない\cite{emotion-hh}.
このように,人間同士の交渉において心理的な効果は交渉結果に影響する重要な要素である.
また,人間とエージェントが交渉を行う際も怒りや喜びによって交渉結果が変化する\cite{emotion-ha}.
前述のようにエージェント同士の交渉に用いるエージェントは心理的要素などを考慮していないため,人間との交渉に用いることは困難である\cite{pinocchio}.

人間とエージェントの交渉は自動交渉と比較すると研究が進んでいなかった.
原因として人間との交渉に最適化されたプラットフォームがなかったというのが考えられる.
人間とエージェントの交渉のためのプラットフォームとして複数論点交渉問題について扱うColored Trails\cite{ct},
Colored TrailsをWEBブラウザで操作できるように拡張したWebCT\cite{webct},自然言語による交渉に焦点を当てたNegoChat\cite{negochat}などがある.
しかし,これらのプラットフォームでは相手と情報を交換するための手段が少ない,感情を表出できないなどの問題がある\cite{pinocchio}.
人間と交渉を行うことができるエージェントを作成するためのプラットフォームとしてIAGO\cite{pinocchio, iago}が登場してからは,
自動交渉エージェント競技会でIAGOを用いた人間とエージェントの交渉を取り扱うリーグが開催されるなど人間とエージェントの交渉に関する研究が進みつつある\cite{hal-2017, hal-2018, hal-2019}.

Koleyら\cite{tki-ha}はTKI\cite{tki}で人間を分類し,特定のモードに分類された場合に譲歩速度を変えるエージェントを提案しており,
合意案の多くがパレート最適な解候補の近辺にあるという交渉結果を得ている.
Koleyらの研究で用いられているエージェントは良い交渉結果をもたらしているが,TKIによる分類の方法に関して問題点もいくつか考えられる.
第一にTKIの特定のモードに対してのみ焦点が当てられているという点である.
TKIの5つのモードのうち妥協と順応の2つに該当した場合は譲歩速度を変更するが,残りの3つのモードの戦略に関しては言及していない.
第二に相手のモードを分類する際に現在の提案で得られる効用と過去にされた提案による効用のみを使用している点である.
人間との交渉では提案以外の行動も行われるため,それらも考慮して分類を行うべきである.
第三にTKIのような分類法では人間を記述することは難しいという点である.
\secref{sec:personality}で詳述するが,人間をカテゴリー化するような方法では個人の特徴を的確に捉えることは困難である.

前述のように人間同士の交渉では感情が重要な役割を持っている.
しかし,自動交渉のエージェントはもちろんであるが,人間との交渉のためのエージェントであっても相手の提案内容のみによって戦略を変更する場合がほとんどである.
そのため,相手の感情など提案以外の行動を考慮せずにエージェントの戦略が考えられているというのが現状である.

\section{本研究の目的}
本稿では,人間と円滑に交渉を行い,信頼関係を築きながら効用を高められるようなエージェントを作成することを目的としている.
具体的には以下の2点である.

\begin{enumerate}
    \item 交渉中の相手の行動や提案内容によって\secref{sec:personality}で詳述するビッグファイブの各因子を測定することを試みる
    \item 測定したビッグファイブの値によって譲歩速度を変更する戦略を提案する
\end{enumerate}

本稿では,TKI,ビッグファイブといった人間のパーソナリティを測るための手法を使用し,相手の特性に合わせて譲歩速度を変化させることで,円滑な交渉を実現する.
また,提案手法および既存研究のエージェントと人間が交渉を行う実験を実施し評価を行うことで,提案手法の有効性を示すことを目的とする.

\section{本論文の構成}
以下に本論文の構成を述べる.第2章では,関連研究として,自動交渉エージェント競技会,IAGO,プロファイリング手法についての概要,
TKIを利用した人間とエージェントの交渉についての研究について述べる.
第3章では,本研究で取り扱う交渉問題である複数論点交渉問題について述べる.
第4章では,提案手法としてプロファイリング手法を用いた交渉戦略について述べる.
第5章では,提案手法で用いるパラメータを調整するために行った予備実験の概要・結果について述べる.
第6章では,予備実験で決定したパラメータを用いて被験者に対して行なった評価実験の概要・結果について述べる.
最後に,第7章では本研究のまとめと今後の課題を示す.

\bibliography{reference}
% 付録用
%\chapter*{付録}


\ifthmaster
  \externals
\begin{externalsenum}{H}
\item \underline{松下昌悟},藤田桂英.
    心理的効果を用いた人間とエージェントの繰り返し交渉戦略.
    電子情報通信学会 人工知能と知識処理研究会(AI), July 2019.
\end{externalsenum}

\fi

\end{document}