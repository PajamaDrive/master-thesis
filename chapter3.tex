\documentclass[a4paper,11pt]{jreport}
\usepackage[master]{tuats}
\usepackage{ascmac}
\usepackage{mathtools}
\usepackage{booktabs}
\debugtrue

% PDF用メタデータ
% コメントアウトすれば著者名やタイトルをPDFに埋められるが,問題が発生しがち
% \AtBeginDvi{
%     \special{pdf:tounicode EUC-UCS2}
%     \special{pdf:docinfo <<
%         /Author (安井貴規 (Takaki YASUI))
%         /Title  (藤田桂英研究室栄光のLaTeXフォーマット (A LaTeX Style Files for Fujita Katsuhide Labratory))
%     >>}
% }

%タイトル
\title{人間とエージェントの交渉における \\ プロファイリングを用いた交渉戦略}
\etitle{Strategy using profiling for human-agent negotiation}

%名前
\author{松下昌悟}
\eauthor{Shogo Matsushita}

%入学年度
\enteryear{2019}
%卒業年度
\graduateyear{2021}

%学籍番号
\studentnumber{19646029}

%提出日
\date{2021年1月X日}
\begin{document}

\chapter{問題設定}
\section{複数論点交渉問題}
本研究では,交渉問題の中でも論点が複数存在する複数論点交渉問題を扱う.エージェント$A_1$と$A_2$が交渉を行う場合を考える.
エージェント$a \in \{ A_1, A_2 \}$の目的関数$f$は,$a$の効用関数$U_{a}$とすべての合意案候補集合$S$を用いると\equref{eq:objectFunc}と表すことができる.

\begin{equation}
  f = \argmax_{s \in S} U_{a}(s)
  \label{eq:objectFunc}
\end{equation}
複数論点交渉問題の場合,目的関数$g$は\equref{eq:socialSurplus}で表され,この目的関数の値は社会的余剰と呼ばれる.
\begin{equation}
  g = \argmax_{s \in S} \sum_{a \in \{ A_1, A_2 \}} U_{a}(s)
  \label{eq:socialSurplus}
\end{equation}

一つの交渉問題はドメインと呼ばれ,論点数$N$のドメインは固有の論点集合$I = \{ i_1, i_2, \ldots, i_N \}$を持つ.
また,論点$i_k \in I$は選択肢集合$V_k = \{ v_{k1}, v_{k2}, \ldots ,v_{kn_k} \}$を持つ.ただし,論点$i_k$の選択肢数を$n_k$と定義する.

各論点$i_k \in I$についてそれぞれ選択肢$v_k \in V_k$を一つずつ選んだものを合意案候補(Bid)と呼び,
$s = (v_1, v_2,\ldots, v_N)$ として表現される.また,すべての合意案候補集合$S$は\equref{eq:Bid}と表すことができる.

\begin{equation}
  S = \{ s = (v_1, v_2, \ldots, v_N) |  v_k \in V_k , 1 \leqq k \leqq N \}
  \label{eq:Bid}
\end{equation}

エージェントは各論点$i_k$について重み$w_k$$(\sum_{k = 1}^N w_k = 1)$および選択肢の評価値$eval(v_k \in V_k)$を持つ.
ただし,$eval(v_k \in V_k)$は最大値が1となるように正規化されているものとする.このとき,エージェントの効用関数$U$は\equref{eq:Utility}となる.
\begin{equation}
  U(s) = \sum_{k = 1}^N w_k \cdot eval(v_k)
  \label{eq:Utility}
\end{equation}
また,各エージェントに対し留保価格(reservation value)が設定される場合がある.
留保価格は合意形成に失敗した際にエージェントが獲得できる効用値である.

\bibliography{reference}
% 付録用
%\chapter*{付録}


\ifthmaster
  \externals
\begin{externalsenum}{H}
\item \underline{松下昌悟},藤田桂英.
    心理的効果を用いた人間とエージェントの繰り返し交渉戦略.
    電子情報通信学会 人工知能と知識処理研究会(AI), July 2019.
\end{externalsenum}

\fi

\end{document}