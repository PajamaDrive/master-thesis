\documentclass[a4paper,11pt]{jreport}
\usepackage[master]{tuats}
\debugtrue

% PDF用メタデータ
% コメントアウトすれば著者名やタイトルをPDFに埋められるが,問題が発生しがち
% \AtBeginDvi{
%     \special{pdf:tounicode EUC-UCS2}
%     \special{pdf:docinfo <<
%         /Author (安井貴規 (Takaki YASUI))
%         /Title  (藤田桂英研究室栄光のLaTeXフォーマット (A LaTeX Style Files for Fujita Katsuhide Labratory))
%     >>}
% }

%タイトル
\title{人間とエージェントの交渉における \\ プロファイリングを用いた交渉戦略}
\etitle{Strategy using profiling for human-agent negotiation}

%名前
\author{松下昌悟}
\eauthor{Shogo Matsushita}

%入学年度
\enteryear{2019}
%卒業年度
\graduateyear{2020}

%学籍番号
\studentnumber{19646029}

%提出日
\date{2021年1月29日}
\begin{document}

\chapter{プロファイリングを用いた交渉戦略}
\section{戦略の概要}
相手の性格や特性を推定し,特性に応じて自分の行動を決定することは交渉を含めた日常生活でも頻繁に行われる.
前述のようにTKIは対立関係に置かれた際の人間の対応方法を分類したものであり,ビッグファイブは5つの因子の値によって人間の特性を表したものである.

本研究では,TKIとビッグファイブを用いて相手の特性を推定し,相手の特性によって戦略を変更するエージェントを提案する.
エージェントは相手の提案や送信された感情・メッセージなどからTKIとビッグファイブを用いて相手の特性を推定する.TKIは相手の提案から推定を行い,ビッグファイブは提案と相手の行動を総合的に判断して各因子の推定を行う.
各因子・カテゴリごとに譲歩速度や行動の頻度などのパラメータを設定しておき,エージェントは相手の特性に応じてこれらのパラメータの値を変更する.

具体的には,TKIでは各カテゴリに分類される確率をそれぞれ計算し,確率を重みとしてパラメータごとに加重平均を計算し,その値を最終的なパラメータの値とする.
ビッグファイブでは各因子の値をそれぞれ算出し,その値によって各因子ごとにパラメータを計算し,それらを平均したものを最終的なパラメータの値とする.
こうしてTKIとビッグファイブそれぞれで計算したパラメータの値をさらに平均することによってエージェントの行動を変更する.

\bibliography{reference}
% 付録用
%\chapter*{付録}


\ifthmaster
  \externals
\begin{externalsenum}{H}
\item \underline{松下昌悟},藤田桂英.
    心理的効果を用いた人間とエージェントの繰り返し交渉戦略.
    電子情報通信学会 人工知能と知識処理研究会(AI), July 2019.
\end{externalsenum}

\fi

\end{document}